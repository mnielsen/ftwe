\documentstyle[12pt]{article}

\begin{document}

Lede:

\textbf{Lots of people hate markets:}

\textbf{Informal description:} What the Fundamental Theorem does is
link two

\textbf{Criticisms of the underlying assumptions:}

\textbf{The invisible hand:}

\textbf{What the theorem says:} If the market is at equilibrium, then
the allocation of goods must be Pareto optimal.

Allocation: Each person and each firm decides how they want to
participate in the market.

Feasibility: There's a constraint, though.  The constraint is that all
produced goods are consumed, and there is no unemployment.  Of course,
this isn't true typically.  But it's often going to be approximately
true.


People can decide to participate in the market in many different ways.
Maybe someone .  Furthermore, they'll have different ways of

What does it mean to be optimal?  It 

Pareto optimality: What does it mean to be Pareto Optimal?  There is
no other set of choices so that ever consumer is at least as well off,
and someone is better off.  Someone's utility is bound to be worse.



How can we formalize how we participate in the economy?  Let's build a
model.  We'll split the world up into two parts: consumers and firms.



\end{document}